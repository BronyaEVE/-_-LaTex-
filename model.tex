\documentclass[12pt, a4paper, oneside, fontset=windows]{ctexart}
\usepackage{amsmath, amsthm, amssymb, appendix, bm, ctex, graphicx, hyperref, mathrsfs, fancyhdr, geometry, float}
\usepackage{cite}
\usepackage{listings}
\usepackage{epstopdf}
\usepackage{booktabs}
\usepackage{multirow}
\usepackage{appendix}
\usepackage{hyperref}
\usepackage{enumerate}
\usepackage{threeparttable}
\usepackage{setspace}
\usepackage[T1]{fontenc}
\usepackage{mathptmx}%Times New Roman字体
\usepackage{titletoc}
\usepackage{fontspec}
\usepackage{pdfpages}
\usepackage{tikz}
\usepackage{etoolbox}
\usepackage{xcolor}
\usepackage{caption}
\usepackage{array}
\usepackage{zhlipsum}

\geometry{left=2.5cm} 
\geometry{right=2.5cm} 
\geometry{top=3.5cm} 
\geometry{bottom=2.5cm} 


\pagenumbering{arabic}%设置页码格式,大写字母标页
\pagestyle{fancy}
\fancyhead{} % 初始化页眉
\fancyhead[CO,CE]{浙江财经大学课程作业}
\fancyfoot{} % 初始化页脚
\fancyfoot[CO,CE]{\thepage}
%\fancyfoot[LO]{}
%\fancyfoot[LE]{}
%\fancyfoot[RO]{}
%\fancyfoot[RE]{}
% 设置页眉页脚横线及样式
%页眉线宽,设为0可以去页眉线
\renewcommand{\headrulewidth}{0.3mm} 
%页脚线宽,设为0可以去页眉线
\renewcommand{\footrulewidth}{0.0mm} 

\title{\textbf{为何浙财没有LaTex模板}}
\author{大鹅}
\date{\today}
\linespread{1.5}
\newtheorem{theorem}{定理}[section]
\newtheorem{definition}[theorem]{定义}
\newtheorem{lemma}[theorem]{引理}
\newtheorem{corollary}[theorem]{推论}
\newtheorem{example}[theorem]{例}
\newtheorem{proposition}[theorem]{命题}
\renewcommand{\abstractname}{\Large\textbf{摘要}}

\begin{document}

\newcommand\dunderline[3][-1pt]{{%
  \setbox0=\hbox{#3}
  \ooalign{\copy0\cr\rule[\dimexpr#1-#2\relax]{\wd0}{#2}}}}


  \begin{titlepage}
    \vspace*{25mm}
    \centering
  
    \includegraphics[scale = 1.1]{title.png}
  
    \vspace{5mm}
  
    \zihao{4}\textbf{{ZHEJIANG UNIVERSITY OF FINANCE AND ECONOMICS}}
  
    \vspace{10mm}
  
    \zihao{0}\textbf{\lishu{课程论文}}
  
    \vspace{5mm}
  
  %  \begin{spacing}{1.2}
  %    \LARGE\selectfont{\textbf{\heiti ——我是小标题}}
  %  \end{spacing}
  
    \vspace{35mm}
  
    \flushleft
    \begin{spacing}{1.3}
    
      \hspace{27mm}\LARGE\selectfont{\textbf{\heiti 课程名称:}\dunderline[-3pt]{1pt}{\makebox[78mm][c]{\kaishu 经济学导论}}}
  
      \hspace{27mm}\LARGE\selectfont{\textbf{\heiti 论文题目:}\dunderline[-3pt]{1pt}{\makebox[78mm][c]{\kaishu 为何浙财没有LaTex模板}}}

      \hspace{27mm}\LARGE\selectfont{\textbf{\heiti 作者姓名:}\dunderline[-3pt]{1pt}{\makebox[78mm][c]{\kaishu 大鹅}}}
  
      \hspace{27mm}\LARGE\selectfont{\textbf{\heiti 专\hspace{14mm}业:}\dunderline[-3pt]{1pt}{\makebox[78mm][c]{\kaishu 经济学}}}
  
      \hspace{27mm}\LARGE\selectfont{\textbf{\heiti 学\hspace{14mm}号:}\dunderline[-3pt]{1pt}{\makebox[78mm][c]{1145141919810}}}
      
      \hspace{27mm}\LARGE\selectfont{\textbf{\heiti 任课教师:}\dunderline[-3pt]{1pt}{\makebox[78mm][c]{\kaishu 萨缪尔森}}}

      \hspace{27mm}\LARGE\selectfont{\textbf{\heiti 完成时间:}\dunderline[-3pt]{1pt}{\makebox[78mm][c]{1919.8.10}}}
        
    \end{spacing}
  
    \vspace{25mm}
  
  \end{titlepage}


\maketitle
\setcounter{page}{0}
\maketitle
\thispagestyle{empty}

\begin{abstract}
    \kaishu 浙财在浙江录取分数线耻辱性的大败已经成为了同学们讨论的话题,
    “输完工商输理工,再接下来输杭师大,接下去没人输了。”一向直言的老学长
    说道。“那么说浙财是备战本降专最早的大学?”“太好了,接下来要输金融职业学院了。”
    \par\textbf{关键词:} \heiti 浙财; 输; 赢; 麻了. 
\end{abstract}

\newpage
\pagenumbering{Roman}
%\setcounter{page}{1}
\tableofcontents
\newpage
%\setcounter{page}{1}
\pagenumbering{arabic}


\section{前言}
这是section\cite{ref1}这是尾注,参考文献用的

\paragraph{}
这是段落,分段用的

\subsection{小标题名字}
这是subsection,分小点用的

\begin{figure}[H]
    \centering
    \includegraphics[scale=0.6]{title.png}
    \caption{学校图标\protect\footnotemark}
    \end{figure} 
\footnotetext{数据来自浙财官网}
标注图片引用、标题用的

\section{后记}
好了,你可以开始写课程论文了,祝好!


\newpage
\bibliography{BE}
\begin{thebibliography}{99}
    \bibitem{ref1}浙财2022分数线
\end{thebibliography}

%\newpage

%\begin{appendices}
    %\renewcommand{\thesection}{\Alph{section}}
    %\section{附录标题}
        %这里是附录. 
%\end{appendices}

\end{document}